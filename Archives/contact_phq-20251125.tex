% Options for packages loaded elsewhere
\PassOptionsToPackage{unicode}{hyperref}
\PassOptionsToPackage{hyphens}{url}
\documentclass[
  english,
  man]{apa6}
\usepackage{xcolor}
\usepackage{amsmath,amssymb}
\setcounter{secnumdepth}{-\maxdimen} % remove section numbering
\usepackage{iftex}
\ifPDFTeX
  \usepackage[T1]{fontenc}
  \usepackage[utf8]{inputenc}
  \usepackage{textcomp} % provide euro and other symbols
\else % if luatex or xetex
  \usepackage{unicode-math} % this also loads fontspec
  \defaultfontfeatures{Scale=MatchLowercase}
  \defaultfontfeatures[\rmfamily]{Ligatures=TeX,Scale=1}
\fi
\usepackage{lmodern}
\ifPDFTeX\else
  % xetex/luatex font selection
\fi
% Use upquote if available, for straight quotes in verbatim environments
\IfFileExists{upquote.sty}{\usepackage{upquote}}{}
\IfFileExists{microtype.sty}{% use microtype if available
  \usepackage[]{microtype}
  \UseMicrotypeSet[protrusion]{basicmath} % disable protrusion for tt fonts
}{}
\makeatletter
\@ifundefined{KOMAClassName}{% if non-KOMA class
  \IfFileExists{parskip.sty}{%
    \usepackage{parskip}
  }{% else
    \setlength{\parindent}{0pt}
    \setlength{\parskip}{6pt plus 2pt minus 1pt}}
}{% if KOMA class
  \KOMAoptions{parskip=half}}
\makeatother
% Make \paragraph and \subparagraph free-standing
\makeatletter
\ifx\paragraph\undefined\else
  \let\oldparagraph\paragraph
  \renewcommand{\paragraph}{
    \@ifstar
      \xxxParagraphStar
      \xxxParagraphNoStar
  }
  \newcommand{\xxxParagraphStar}[1]{\oldparagraph*{#1}\mbox{}}
  \newcommand{\xxxParagraphNoStar}[1]{\oldparagraph{#1}\mbox{}}
\fi
\ifx\subparagraph\undefined\else
  \let\oldsubparagraph\subparagraph
  \renewcommand{\subparagraph}{
    \@ifstar
      \xxxSubParagraphStar
      \xxxSubParagraphNoStar
  }
  \newcommand{\xxxSubParagraphStar}[1]{\oldsubparagraph*{#1}\mbox{}}
  \newcommand{\xxxSubParagraphNoStar}[1]{\oldsubparagraph{#1}\mbox{}}
\fi
\makeatother
\usepackage{graphicx}
\makeatletter
\newsavebox\pandoc@box
\newcommand*\pandocbounded[1]{% scales image to fit in text height/width
  \sbox\pandoc@box{#1}%
  \Gscale@div\@tempa{\textheight}{\dimexpr\ht\pandoc@box+\dp\pandoc@box\relax}%
  \Gscale@div\@tempb{\linewidth}{\wd\pandoc@box}%
  \ifdim\@tempb\p@<\@tempa\p@\let\@tempa\@tempb\fi% select the smaller of both
  \ifdim\@tempa\p@<\p@\scalebox{\@tempa}{\usebox\pandoc@box}%
  \else\usebox{\pandoc@box}%
  \fi%
}
% Set default figure placement to htbp
\def\fps@figure{htbp}
\makeatother
% definitions for citeproc citations
\NewDocumentCommand\citeproctext{}{}
\NewDocumentCommand\citeproc{mm}{%
  \begingroup\def\citeproctext{#2}\cite{#1}\endgroup}
\makeatletter
 % allow citations to break across lines
 \let\@cite@ofmt\@firstofone
 % avoid brackets around text for \cite:
 \def\@biblabel#1{}
 \def\@cite#1#2{{#1\if@tempswa , #2\fi}}
\makeatother
\newlength{\cslhangindent}
\setlength{\cslhangindent}{1.5em}
\newlength{\csllabelwidth}
\setlength{\csllabelwidth}{3em}
\newenvironment{CSLReferences}[2] % #1 hanging-indent, #2 entry-spacing
 {\begin{list}{}{%
  \setlength{\itemindent}{0pt}
  \setlength{\leftmargin}{0pt}
  \setlength{\parsep}{0pt}
  % turn on hanging indent if param 1 is 1
  \ifodd #1
   \setlength{\leftmargin}{\cslhangindent}
   \setlength{\itemindent}{-1\cslhangindent}
  \fi
  % set entry spacing
  \setlength{\itemsep}{#2\baselineskip}}}
 {\end{list}}
\usepackage{calc}
\newcommand{\CSLBlock}[1]{\hfill\break\parbox[t]{\linewidth}{\strut\ignorespaces#1\strut}}
\newcommand{\CSLLeftMargin}[1]{\parbox[t]{\csllabelwidth}{\strut#1\strut}}
\newcommand{\CSLRightInline}[1]{\parbox[t]{\linewidth - \csllabelwidth}{\strut#1\strut}}
\newcommand{\CSLIndent}[1]{\hspace{\cslhangindent}#1}
\ifLuaTeX
\usepackage[bidi=basic]{babel}
\else
\usepackage[bidi=default]{babel}
\fi
% get rid of language-specific shorthands (see #6817):
\let\LanguageShortHands\languageshorthands
\def\languageshorthands#1{}
\ifLuaTeX
  \usepackage[english]{selnolig} % disable illegal ligatures
\fi
\setlength{\emergencystretch}{3em} % prevent overfull lines
\providecommand{\tightlist}{%
  \setlength{\itemsep}{0pt}\setlength{\parskip}{0pt}}
% Manuscript styling
\usepackage{upgreek}
\captionsetup{font=singlespacing,justification=justified}

% Table formatting
\usepackage{longtable}
\usepackage{lscape}
% \usepackage[counterclockwise]{rotating}   % Landscape page setup for large tables
\usepackage{multirow}		% Table styling
\usepackage{tabularx}		% Control Column width
\usepackage[flushleft]{threeparttable}	% Allows for three part tables with a specified notes section
\usepackage{threeparttablex}            % Lets threeparttable work with longtable

% Create new environments so endfloat can handle them
% \newenvironment{ltable}
%   {\begin{landscape}\centering\begin{threeparttable}}
%   {\end{threeparttable}\end{landscape}}
\newenvironment{lltable}{\begin{landscape}\centering\begin{ThreePartTable}}{\end{ThreePartTable}\end{landscape}}

% Enables adjusting longtable caption width to table width
% Solution found at http://golatex.de/longtable-mit-caption-so-breit-wie-die-tabelle-t15767.html
\makeatletter
\newcommand\LastLTentrywidth{1em}
\newlength\longtablewidth
\setlength{\longtablewidth}{1in}
\newcommand{\getlongtablewidth}{\begingroup \ifcsname LT@\roman{LT@tables}\endcsname \global\longtablewidth=0pt \renewcommand{\LT@entry}[2]{\global\advance\longtablewidth by ##2\relax\gdef\LastLTentrywidth{##2}}\@nameuse{LT@\roman{LT@tables}} \fi \endgroup}

% \setlength{\parindent}{0.5in}
% \setlength{\parskip}{0pt plus 0pt minus 0pt}

% Overwrite redefinition of paragraph and subparagraph by the default LaTeX template
% See https://github.com/crsh/papaja/issues/292
\makeatletter
\renewcommand{\paragraph}{\@startsection{paragraph}{4}{\parindent}%
  {0\baselineskip \@plus 0.2ex \@minus 0.2ex}%
  {-1em}%
  {\normalfont\normalsize\bfseries\itshape\typesectitle}}

\renewcommand{\subparagraph}[1]{\@startsection{subparagraph}{5}{1em}%
  {0\baselineskip \@plus 0.2ex \@minus 0.2ex}%
  {-\z@\relax}%
  {\normalfont\normalsize\itshape\hspace{\parindent}{#1}\textit{\addperi}}{\relax}}
\makeatother

\makeatletter
\usepackage{etoolbox}
\patchcmd{\maketitle}
  {\section{\normalfont\normalsize\abstractname}}
  {\section*{\normalfont\normalsize\abstractname}}
  {}{\typeout{Failed to patch abstract.}}
\patchcmd{\maketitle}
  {\section{\protect\normalfont{\@title}}}
  {\section*{\protect\normalfont{\@title}}}
  {}{\typeout{Failed to patch title.}}
\makeatother

\usepackage{xpatch}
\makeatletter
\xapptocmd\appendix
  {\xapptocmd\section
    {\addcontentsline{toc}{section}{\appendixname\ifoneappendix\else~\theappendix\fi: #1}}
    {}{\InnerPatchFailed}%
  }
{}{\PatchFailed}
\makeatother
\keywords{police stop; psychopathology; black; race; legal system expsoure\newline\indent Word count: 1264}
\DeclareDelayedFloatFlavor{ThreePartTable}{table}
\DeclareDelayedFloatFlavor{lltable}{table}
\DeclareDelayedFloatFlavor*{longtable}{table}
\makeatletter
\renewcommand{\efloat@iwrite}[1]{\immediate\expandafter\protected@write\csname efloat@post#1\endcsname{}}
\makeatother
\usepackage{lineno}

\linenumbers
\usepackage{csquotes}
\usepackage{bookmark}
\IfFileExists{xurl.sty}{\usepackage{xurl}}{} % add URL line breaks if available
\urlstyle{same}
\hypersetup{
  pdftitle={Police stop and depressive symptoms: Examining moderating role of race},
  pdfauthor={Mohammad Hahsim1},
  pdflang={en-EN},
  pdfkeywords={police stop; psychopathology; black; race; legal system expsoure},
  hidelinks,
  pdfcreator={LaTeX via pandoc}}

\title{Police stop and depressive symptoms: Examining moderating role of race}
\author{Mohammad Hahsim\textsuperscript{1}}
\date{}


\shorttitle{Police stop and depressive symptoms}

\authornote{

A sample project for the course of Graduate Seminar in Psychology, led by Dr.~Moin Syed.

The authors made the following contributions. Mohammad Hahsim: Conceptualization, Writing - Original Draft Preparation, Writing - Review \& Editing.

Correspondence concerning this article should be addressed to Mohammad Hahsim, Department of Psychology. E-mail: \href{mailto:hashi205@umn.edu}{\nolinkurl{hashi205@umn.edu}}

}

\affiliation{\vspace{0.5cm}\textsuperscript{1} University of Minnesota, Twin Cities}

\abstract{%
Police stops are increasingly recognized as psychologically consequential events that may elevate depressive symptoms, particularly among marginalized groups. The present study used a simulated dataset of 500 participants to examine whether experiencing a police stop was associated with higher depressive symptoms, and whether this association was moderated by race. Participants ranged from early to late adolescence (M = 27.52, SD = 0.42) and were demographically diverse: 43.40\% identified as female, 58.80\% identified as BIPOC, and 27\% reported negative police contact. Depressive symptoms were assessed using the PHQ-9. Analyses were conducted in R and proceeded in two steps. First, a Welch t-test revealed that individuals who had been stopped by the police reported significantly higher depressive symptoms than those who had not, t(255.31) = -12.10, p \textless{} .001. Second, a linear regression model tested whether race moderated this association. Although the model explained a significant portion of variance in depressive symptoms (\(R^2\) = .26), the police contact × race interaction was not significant, B = -0.13, p = .938, indicating that the psychological impact of police stops did not differ between White and BIPOC participants. Together, findings from this simulated dataset suggest that police contact is strongly associated with elevated depressive symptoms, but this association appears consistent across racial groups.
}



\begin{document}
\maketitle

Police interactions, especially involuntary or intrusive stops, are increasingly recognized as significant stressors that may undermine mental health (J. DeVylder, Fedina, \& Link, 2020). A growing body of research shows that being stopped by the police can evoke fear, threat, and feelings of injustice(Jackson Davis, 2022), all of which may contribute to elevated depressive symptoms (Harris \& Cortés, 2022) . However, the psychological impact of police contact is not experienced uniformly across communities(Jackson Davis, 2022). Race remains a central factor shaping how individuals perceive, interpret, and internalize police encounters(Harris, 2025; Jackson, Fix, et al., 2025). For many racial and ethnic minority groups, especially Black and Latino communities(Briere \& Runtz, 2024), police stops occur within a broader historical and social context marked by discrimination and disproportionate surveillance(Del Toro et al., 2019). The present study examines the association between police stops and depressive symptoms and investigates whether this relationship differs by race. Understanding racial variation in the mental health consequences of police contact is essential for clarifying risk pathways and identifying populations most adversely affected(J. E. DeVylder, Anglin, Bowleg, Fedina, \& Link, 2022). This work contributes to ongoing discussions on policing, public health, and racial inequality by evaluating whether race moderates the psychological burden of police stops.

\section{Methods}\label{methods}

The current study was \emph{NOT} preregistered. Data and code are available at \url{https://github.com/hashi205-ship-it/contact_phq}. The study uses a simulated dataset generated for teaching and learning purposes.

\subsection{Participants}\label{participants}

The present study uses a simulated dataset comprising 500 participants. Participants ranged in age from early to late adolescence, with a mean age of 27.52 years (SD = 0.42). The sample was demographically diverse. Approximately 43.40\% of the sample identified as female, and 58.80\% identified as belonging to a BIPOC racial or ethnic group. 27\% particapants had negative police contact. In addition, 50\% of participants were immigrants and rest were non-immigrants.

\subsection{Measures}\label{measures}

\subsubsection{Police Contact}\label{police-contact}

Participants self reported whether they had been stopped by the police in yes or no responses. This direct question approach has previously been used in the literature(Jackson, Qureshi, Testa, \& Prins, 2025).

\subsubsection{Depressive Symptom}\label{depressive-symptom}

Participants completed the Patient Health Questionnaire--9 (Kroenke, Spitzer, \& Williams, 2001), a widely used and well-validated self-report measure of depressive symptomatology. The PHQ-9 assesses the frequency of nine DSM-based symptoms of major depression experienced over the past two weeks (e.g., anhedonia, depressed mood, sleep disturbance, fatigue, and difficulty concentrating). Items are rated on a 4-point Likert scale ranging from 0 (not at all) to 3 (nearly every day), with total scores reflecting overall severity of depressive symptoms. Higher scores indicate greater depressive symptom severity, with established clinical cutoffs corresponding to mild, moderate, moderately severe, and severe depression. In the present sample, the PHQ-9 demonstrated excellent internal consistency (Cronbach's alpha = 0.95), consistent with prior research supporting its reliability and construct validity.

\subsection{Procedure}\label{procedure}

All data were simulated to approximate realistic distributions. Participants hypothetically reported demographics, police contact, and depressive symptoms.

\subsection{Data Analysis}\label{data-analysis}

Data analyses were conducted in R using tidyverse, psych, emmeans, and ggplot2 packages. All analyses were performed on the simulated dataset after computing PHQ-9 total scores by summing the nine individual symptom items. Prior to analysis, categorical predictors were coded as factors with meaningful reference categories (i.e., No for police contact and BIPOC for race) to facilitate interpretation of regression coefficients.

Analyses proceeded in two steps. First, to evaluate Hypothesis first, which predicted that individuals who had been stopped by the police would report higher depressive symptoms than those who had not, we conducted a Welch two-sample t-test comparing PHQ-9 total scores across police contact groups (``Yes'' vs.~``No''). This test allowed for unequal variances between groups and provided an estimate of whether depressive symptom severity differed as a function of police contact.

Second, to evaluate second hypothesis, which predicted that race would moderate the association between police contact and depressive symptoms, we estimated a linear regression model including police contact, race, and their interaction term. This moderation model tested whether the effect of police contact on depressive symptoms differed between White and BIPOC participants. Model fit was evaluated using R² and F-tests, and significance of individual predictors was assessed using t-tests with associated confidence intervals. To aid interpretation of the interaction, estimated marginal means were computed using the emmeans package, and a corresponding moderation plot was produced to visualize predicted depressive symptoms across police contact status for each racial group.

All statistical tests used a significance threshold of \(\alpha\) = .05 (two-tailed), and effect sizes and predicted values were reported where relevant. Confidence intervals were computed using model-based standard errors.

\section{Results}\label{results}

\subsection{Descriptive Statistics}\label{descriptive-statistics}

Descriptive analyses were conducted to characterize overall depressive symptom severity and patterns across key demographic and experiential groups. The mean depressive symptom score for the full sample was 28.42 (SD = 6.69), indicating generally moderate levels of depressive symptoms in this simulated dataset. Depressive symptoms differed meaningfully across participants based on police contact. Those who reported being stopped by the police had a notably higher mean PHQ score (33.74), whereas individuals with no history of police stops showed a substantially lower average score (26.46). Differences also emerged at the descriptive level across racial groups. BIPOC participants reported a higher mean level of depressive symptoms (29.37) compared with White participants (26.94). Finally, depressive symptoms varied modestly by immigrant status. Immigrant participants had an average PHQ-9 score of 28.87, slightly higher than the mean for non-immigrant participants (27.92).

\subsection{Inferential Statistics}\label{inferential-statistics}

\subsubsection{Group Differences in Depressive Symptoms by Police Contact}\label{group-differences-in-depressive-symptoms-by-police-contact}

To test the hypothesis that individuals who had experienced a police stop would report higher depressive symptoms than those who had not, a Welch two-sample t-test compared depressive symptom scores across police contact groups. The analysis revealed a large and statistically significant difference in depressive symptoms,
t(255.31) = -12.10, p \textless{} .001. The 95\% confidence interval for the mean difference ranged from -8.47 to -6.10, indicating that individuals with police contact consistently exhibited more severe depressive symptoms. These results provide strong support for first question, suggesting that involuntary police encounters are associated with heightened psychological distress in this sample.

\subsubsection{Moderation by Race}\label{moderation-by-race}

To examine whether race moderated the association between police contact and depressive symptoms, a linear regression model was estimated including the main effects of police contact and race, along with their interaction. The full model accounted for a significant proportion of variance in depressive symptoms, \(R^2\) = .26, adjusted \(R^2\) = .25, F(3, 265) = 31.52, \(p\) \textless{} .001.

For the reference category of BIPOC individuals with no police contact, the estimated mean depressive score was 27.01. A significant main effect of police contact emerged: BIPOC individuals who had been stopped by the police scored, on average, 7.07 points higher on depressive symptoms than their BIPOC counterparts who had not been stopped, B = 7.07, SE = 0.93, t(265) = 7.57, \(p\) \textless{} .001. The main effect of race was not statistically significant, B = -1.19, SE = 0.83, t(265) = -1.43, p = .153, indicating that White and BIPOC participants did not significantly differ in depressive symptoms when they had not been stopped by the police.

Critically, the interaction between police contact and race was not significant, B = -0.13, SE = 1.67, t(265) = -0.08, p = .938. This indicates that the effect of being stopped by the police on depressive symptoms did not differ meaningfully between White and BIPOC participants. Although the descriptive plot suggested slightly higher predicted scores for BIPOC youth following a police stop, this difference was not statistically reliable.

Together, the results show that while police contact is strongly associated with higher depressive symptoms, the magnitude of this association is similar for White and BIPOC participants, providing no evidence in this dataset that race moderates the psychological impact of police stops.

\begin{table}[tbp]

\begin{center}
\begin{threeparttable}

\caption{\label{tab:unnamed-chunk-1}Descriptives Statistics for PHQ scale by police contact}

\begin{tabular}{lll}
\toprule
Police Contact & Mean & SD\\
\midrule
No & 26.46 & 6.45\\
Yes & 33.74 & 3.38\\
NA & 26.37 & 7.29\\
\bottomrule
\addlinespace
\end{tabular}

\begin{tablenotes}[para]
\normalsize{\textit{Note.} The groups significantly differd}
\end{tablenotes}

\end{threeparttable}
\end{center}

\end{table}

\begin{table}[tbp]

\begin{center}
\begin{threeparttable}

\caption{\label{tab:unnamed-chunk-2}Model 2: Moderation by race}

\begin{tabular}{lllll}
\toprule
term & \multicolumn{1}{c}{estimate} & \multicolumn{1}{c}{std.error} & \multicolumn{1}{c}{statistic} & \multicolumn{1}{c}{p.value}\\
\midrule
(Intercept) & 27.01 & 0.57 & 47.60 & 0.00\\
pconYes & 7.07 & 0.93 & 7.57 & 0.00\\
raceWhite & -1.19 & 0.83 & -1.43 & 0.15\\
pconYes:raceWhite & -0.13 & 1.67 & -0.08 & 0.94\\
\bottomrule
\end{tabular}

\end{threeparttable}
\end{center}

\end{table}

\pandocbounded{\includegraphics[keepaspectratio]{contact_phq-20251125_files/figure-latex/plot generation-1.pdf}}
\textbf{Figure 1: Moderation plot for Race}

\section{Discussion}\label{discussion}

No discussion section is written as we used simulated data.
\newpage

\section{References}\label{references}

\phantomsection\label{refs}
\begin{CSLReferences}{1}{0}
\bibitem[\citeproctext]{ref-briere2024}
Briere, J., \& Runtz, M. (2024). Police in the rearview mirror: Social marginalization, trauma, and fear of being killed. \emph{American Journal of Orthopsychiatry}, \emph{94}(1), 15--22. \url{https://doi.org/10.1037/ort0000700}

\bibitem[\citeproctext]{ref-deltoro2019}
Del Toro, J., Lloyd, T., Buchanan, K. S., Robins, S. J., Bencharit, L. Z., Smiedt, M. G., \ldots{} Goff, P. A. (2019). The criminogenic and psychological effects of police stops on adolescent black and Latino boys. \emph{Proceedings of the National Academy of Sciences of the United States of America}, \emph{116}(17), 8261--8268. \url{https://doi.org/10.1073/pnas.1808976116}

\bibitem[\citeproctext]{ref-devylder2022}
DeVylder, J. E., Anglin, D. M., Bowleg, L., Fedina, L., \& Link, B. G. (2022). Police Violence and Public Health. \emph{Annual review of clinical psychology}, \emph{18}, 527--552. \url{https://doi.org/10.1146/annurev-clinpsy-072720-020644}

\bibitem[\citeproctext]{ref-devylder2020}
DeVylder, J., Fedina, L., \& Link, B. (2020). Impact of Police Violence on Mental Health: A Theoretical Framework. \emph{American journal of public health}, \emph{110}(11), 1704--1710. \url{https://doi.org/10.2105/AJPH.2020.305874}

\bibitem[\citeproctext]{ref-harrisPoliceViolenceExposure2025}
Harris, L. K. (2025). \emph{Police violence exposure and cardiometabolic risk in black women} (PhD thesis). United States {\textendash} North Carolina. Retrieved from \url{http://login.ezproxy.lib.umn.edu/login?url=https://www.proquest.com/dissertations-theses/police-violence-exposure-cardiometabolic-risk/docview/3205838164/se-2?accountid=14586}

\bibitem[\citeproctext]{ref-harris2022}
Harris, L. K., \& Cortés, Y. I. (2022). Police Violence and Black Women's Health. \emph{The journal for nurse practitioners : JNP}, \emph{18}(5), 589--590. \url{https://doi.org/10.1016/j.nurpra.2022.02.014}

\bibitem[\citeproctext]{ref-jackson2025}
Jackson, D. B., Fix, R. L., Testa, A., Webb, L., Mendelson, T., Alang, S., \& Bowleg, L. (2025). Police Avoidance Among Black Youth. \emph{Academic pediatrics}, \emph{25}(2), 102594. \url{https://doi.org/10.1016/j.acap.2024.10.006}

\bibitem[\citeproctext]{ref-jackson2025a}
Jackson, D. B., Qureshi, F., Testa, A., \& Prins, S. J. (2025). Police Contact and the Mental Health of Young Adults in the United States. \emph{Journal of Adolescent Health}, \emph{76}(5), 813--820. \url{https://doi.org/10.1016/j.jadohealth.2025.01.015}

\bibitem[\citeproctext]{ref-jacksondavisBlackFirstGenerationUnderresourced2022}
Jackson Davis, A. (2022). \emph{Black, first-generation, underresourced college students: Fighting the dual pandemics of COVID-19 and police brutality} (PhD thesis). United States {\textendash} California. Retrieved from \url{http://login.ezproxy.lib.umn.edu/login?url=https://www.proquest.com/dissertations-theses/black-first-generation-underresourced-college/docview/2705675004/se-2?accountid=14586}

\bibitem[\citeproctext]{ref-kroenke2001}
Kroenke, K., Spitzer, R. L., \& Williams, J. B. W. (2001). The PHQ-9. \emph{Journal of General Internal Medicine}, \emph{16}(9), 606--613. \url{https://doi.org/10.1046/j.1525-1497.2001.016009606.x}

\end{CSLReferences}


\end{document}
